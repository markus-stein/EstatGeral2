% Options for packages loaded elsewhere
\PassOptionsToPackage{unicode}{hyperref}
\PassOptionsToPackage{hyphens}{url}
%
\documentclass[
]{article}
\usepackage{amsmath,amssymb}
\usepackage{lmodern}
\usepackage{ifxetex,ifluatex}
\ifnum 0\ifxetex 1\fi\ifluatex 1\fi=0 % if pdftex
  \usepackage[T1]{fontenc}
  \usepackage[utf8]{inputenc}
  \usepackage{textcomp} % provide euro and other symbols
\else % if luatex or xetex
  \usepackage{unicode-math}
  \defaultfontfeatures{Scale=MatchLowercase}
  \defaultfontfeatures[\rmfamily]{Ligatures=TeX,Scale=1}
\fi
% Use upquote if available, for straight quotes in verbatim environments
\IfFileExists{upquote.sty}{\usepackage{upquote}}{}
\IfFileExists{microtype.sty}{% use microtype if available
  \usepackage[]{microtype}
  \UseMicrotypeSet[protrusion]{basicmath} % disable protrusion for tt fonts
}{}
\makeatletter
\@ifundefined{KOMAClassName}{% if non-KOMA class
  \IfFileExists{parskip.sty}{%
    \usepackage{parskip}
  }{% else
    \setlength{\parindent}{0pt}
    \setlength{\parskip}{6pt plus 2pt minus 1pt}}
}{% if KOMA class
  \KOMAoptions{parskip=half}}
\makeatother
\usepackage{xcolor}
\IfFileExists{xurl.sty}{\usepackage{xurl}}{} % add URL line breaks if available
\IfFileExists{bookmark.sty}{\usepackage{bookmark}}{\usepackage{hyperref}}
\hypersetup{
  pdftitle={Plano Aula 09 e 10},
  pdfauthor={Markus Stein},
  hidelinks,
  pdfcreator={LaTeX via pandoc}}
\urlstyle{same} % disable monospaced font for URLs
\usepackage[margin=1in]{geometry}
\usepackage{graphicx}
\makeatletter
\def\maxwidth{\ifdim\Gin@nat@width>\linewidth\linewidth\else\Gin@nat@width\fi}
\def\maxheight{\ifdim\Gin@nat@height>\textheight\textheight\else\Gin@nat@height\fi}
\makeatother
% Scale images if necessary, so that they will not overflow the page
% margins by default, and it is still possible to overwrite the defaults
% using explicit options in \includegraphics[width, height, ...]{}
\setkeys{Gin}{width=\maxwidth,height=\maxheight,keepaspectratio}
% Set default figure placement to htbp
\makeatletter
\def\fps@figure{htbp}
\makeatother
\setlength{\emergencystretch}{3em} % prevent overfull lines
\providecommand{\tightlist}{%
  \setlength{\itemsep}{0pt}\setlength{\parskip}{0pt}}
\setcounter{secnumdepth}{-\maxdimen} % remove section numbering
\usepackage{fancyhdr}
\ifluatex
  \usepackage{selnolig}  % disable illegal ligatures
\fi

\title{Plano Aula 09 e 10}
\author{Markus Stein}
\date{}

\begin{document}
\maketitle

\addtolength{\headheight}{1.0cm} 
\pagestyle{fancyplain} 
\lhead{\includegraphics[height=1.5cm]{LogoIME.png}}
\rhead{\includegraphics[height=1.5cm]{images.jpeg}}
\chead{UNIVERSIDADE FEDERAL DO RIO GRANDE DO SUL \\
INSTITUTO DE MATEMÁTICA E ESTATÍSTICA \\
DEPARTAMENTO DE ESTATÍSTICA \\
\vspace{0.3cm}
MAT02215 - Estatística Geral 2 - 2021/1
}
\renewcommand{\headrulewidth}{0pt}

\hypertarget{continuauxe7uxe3o-intervalos-de-confianuxe7a}{%
\subsection{(\ldots continuação) Intervalos de
Confiança}\label{continuauxe7uxe3o-intervalos-de-confianuxe7a}}

Já vimos até aqui

\begin{itemize}
\tightlist
\item
  IC para uma média populacional \(\mu\)

  \begin{itemize}
  \tightlist
  \item
    supondo \(\sigma^2\) conhecido (ou \(n>30\)),
  \item
    supondo \(\sigma^2\) desconhecido (e \(n<30\)) e\\
  \end{itemize}
\item
  IC para diferença de médias \(\mu_1 - \mu_2\).
\end{itemize}

\vspace{0.5cm}

\hypertarget{intervalo-de-confianuxe7a-para-a-variuxe2ncia}{%
\subsection{Intervalo de confiança para a
Variância}\label{intervalo-de-confianuxe7a-para-a-variuxe2ncia}}

\begin{itemize}
\item
  Suponha que agora queremos estimar uma variância populacional
  \(\sigma^2\).
\item
  Exemplo: Estimar a variabilidade dos retornos de certa aplicação
  financeira.

  \begin{itemize}
  \tightlist
  \item
    Qual o estimador pontual ``natural'' para o problema? E como
    calcular um IC para \(\sigma^2\)?
  \end{itemize}
\end{itemize}

\hypertarget{continuauxe7uxe3o-estimauxe7uxe3o-de-sigma2}{%
\subsubsection{\texorpdfstring{(\ldots continuação) Estimação de
\(\sigma^2\)}{(\ldots continuação) Estimação de \textbackslash sigma\^{}2}}\label{continuauxe7uxe3o-estimauxe7uxe3o-de-sigma2}}

\begin{itemize}
\item
  Se desconhecemos a variância populacional, podemos estimá-la usando o
  estimador \(S^2 = \frac{\sum_{i=1}^n (X_i - \overline X)^2}{n-1}\)
  (porquê?)
\item
  Nesse caso \(S^2\) é uma variável aleatória (v.a.). (Sabemos qual a
  distribuição amostral de \(S^2\)?)
\end{itemize}

\vspace{0.5cm}

\hypertarget{distribuiuxe7uxe3o-de-probabilidade-qui-quadrado-bussab-e-morettin---puxe1g.-358}{%
\subsubsection{\texorpdfstring{Distribuição (de probabilidade)
\(Qui-Quadrado\) (Bussab e Morettin - pág.
358)}{Distribuição (de probabilidade) Qui-Quadrado (Bussab e Morettin - pág. 358)}}\label{distribuiuxe7uxe3o-de-probabilidade-qui-quadrado-bussab-e-morettin---puxe1g.-358}}

Teorema (\textbf{Distribuição Qui-Quadrado, nossa versão}): Seja
\(X_1, \ldots, X_n\) uma amostra aleatória da v.a.
\(X \sim Normal(\mu,\sigma^2)\) e
\(S^2 = \sum_{i=1}^n (X_i - \overline X)^2 / (n-1)\), então podemos
escrever uma quantidade \(Q\) tal que (dadas algumas outras suposições
que omitimos aqui)
\[ Q = \frac{(n-1) S^2}{\sigma^2} \sim \chi^2_{(n-1)}.\] em que
\(\chi^2_{(n-1)}\) denota a distribuição de probabilidade Qui-Quadrado
com \(n-1\) graus de liberdade (g.l.).

\begin{itemize}
\item
  A distribuição \(\chi^2\) possui \textbf{valores tabelados}, assim
  como a distribuição \textbf{normal padrão} e a \(t\). A diferença é
  que \(Q\) só assume valores positivos.
\item
  Como usar a distribuição de \(Q\) para construir um IC para
  \(\sigma^2\)? \textbf{Quais as suposições necessárias}? \textbf{Como
  interpretar os resultados}?
\end{itemize}

\vspace{0.5cm}

\clearpage

\hypertarget{intervalo-para-uma-proporuxe7uxe3o-populacional}{%
\subsection{Intervalo para uma proporção
(populacional)}\label{intervalo-para-uma-proporuxe7uxe3o-populacional}}

\begin{itemize}
\item
  Suponha que agora queremos estimar uma proporção populacional \(\pi\).
\item
  Exemplo: Estimar a proporção de pessoas infectadas por um certo vírus
  numa população.

  \begin{itemize}
  \tightlist
  \item
    Qual o estimador pontual ``natural'' para o problema? E como
    calcular um IC para \(\pi\)?\\
  \end{itemize}
\item
  \textbf{Quais as suposições necessárias}? \textbf{Como interpretar os
  resultados}?
\end{itemize}

\vspace{0.5cm}

\hypertarget{usando-o-teorema-central-do-limite}{%
\subsubsection{Usando o teorema central do
limite}\label{usando-o-teorema-central-do-limite}}

\begin{itemize}
\tightlist
\item
  \(\frac{\overline X - \mu}{\sigma / \sqrt{n}} \sim Normal(0,1)\) se
  \(X \sim Normal(\mu, \sigma^2)\), para \(\sigma^2\) conhecido, ou
\item
  \(\frac{\overline X - \mu}{S / \sqrt{n}} \sim Normal(0,1)\) se o
  tamanho amostral for grande, \(n >> 30\).
\end{itemize}

\textbf{No caso da proporção amostral \(X\) não será normal}

Para uma amostra aleatória \(X_1, \ldots, X_n\) da v.a.
\(X \sim Bernoulli(\pi)\) temos que
\(\sum_{i=1}^n X_i \sim Binomial(n, \pi)\). Das propriedades da
distribuição binomial sabemos que \(E(\sum_{i=1}^n X_i) = np\) e
\(V(\sum_{i=1}^n X_i) = np(1-p)\).

Assim, para um tamanho de amostra suficientemente grande (\(n >> 30\))
\[Z =  \frac{(\sum_{i=1}^n X_i) - np}{\sqrt{np(1-p)}} \sim Normal(0,1)\]
ou ainda usando \(p = \sum_{i=1}^n X_i / n\)
\[Z =  \frac{(\sum_{i=1}^n X_i/n) - p}{\sqrt{\frac{p(1-p)}{n}}} \sim Normal(0,1)\]

\vspace{0.5cm}

\hypertarget{dimensionamento-de-amostra}{%
\subsection{Dimensionamento de
amostra}\label{dimensionamento-de-amostra}}

Chamamos de erro de estimação a metade da amplitude do intervalo, * no
caso de IC para \(\mu\) com \(\sigma^2\) conhecido,
\(E = z_{\alpha/2} \times \sigma / \sqrt{n}\),\\
* no caso de IC para \(\mu\) com \(\sigma^2\) desconhecido e \(n\)
pequeno, \(E = t_{(n-1); \alpha/2} \times s / \sqrt{n}\),\\
* e no caso de IC para \(\pi\), \(E = z_{\alpha/2} * \sqrt{p(1-p)/n}\).

Como calcular o tamanho mínimo de uma amostra para uma confiança
\(1-\alpha\) especificada e um erro máximo \(E\) também fixado?

\vspace{0.5cm}

\begin{center}\rule{0.5\linewidth}{0.5pt}\end{center}

\hypertarget{ler-slides-das-aulas-9-e-10}{%
\subsubsection{Ler slides das aulas 9 e
10}\label{ler-slides-das-aulas-9-e-10}}

\hypertarget{fazer-exercuxedcios-continuauxe7uxe3o-da-lista-1-3}{%
\subsubsection{Fazer exercícios continuação da lista
1-3}\label{fazer-exercuxedcios-continuauxe7uxe3o-da-lista-1-3}}

\hypertarget{fazer-avaliauxe7uxe3o-parcial-da-uxe1rea-1-prova-1---vale-nota}{%
\subsubsection{Fazer avaliação parcial da área 1 (PROVA 1) - vale
nota!!!}\label{fazer-avaliauxe7uxe3o-parcial-da-uxe1rea-1-prova-1---vale-nota}}

\begin{center}\rule{0.5\linewidth}{0.5pt}\end{center}

\vspace{0.5cm}

\hypertarget{rever-para-a-prova-1}{%
\subsubsection{Rever para a prova 1}\label{rever-para-a-prova-1}}

\begin{itemize}
\tightlist
\item
  Introdução e definições; amostagem.
\item
  Estimação pontual; parâmetros e estimadores. erro padrão.
\item
  Propriedades dos estimadores; viés, eficiência e consistência; EQM.
\item
  Distribuições amostrais; TCL.
\item
  Intervalos de confiança: para uma e duas médias, uma variância e uma
  proporção.
\item
  Interpretação e suposições para intervalors de confiança.
\item
  Erro de estimação e tamanho de amostra.
\end{itemize}

\end{document}
